% Tipo de documento e pacotes utilizados:
\documentclass[conference]{IEEEtran}

\IEEEoverridecommandlockouts
% The preceding line is only needed to identify funding in the first footnote. If that is unneeded, please comment it out.

\usepackage{algorithmic}
\usepackage{amsmath,amssymb,amsfonts}
\usepackage{cite}
\usepackage{graphicx}
\usepackage{textcomp}
\usepackage{xcolor}

% Pacotes adicionais para documentos em português:
\usepackage[portuguese]{babel}
\usepackage[T1]{fontenc}
\usepackage[utf8]{inputenc}

% Definição de BibTex:
\def\BibTeX{{\rm B\kern-.05em{\sc i\kern-.025em b}\kern-.08em
    T\kern-.1667em\lower.7ex\hbox{E}\kern-.125emX}}
    
% Documento:
\begin{document}

\title{Experimento 2 - Regulador de Tensão\\
\break Pré-relatório}

\author{\IEEEauthorblockN{Victor André Gris Costa}
\IEEEauthorblockA{\textit{Universidade de Brasília - UnB} \\
\textit{Departamento de Engenharia Elétrica}\\
Brasília, Brasil \\
victorandr98@gmail.com}
}
\maketitle

\begin{abstract}
Este é o pré-relatório para o experimento 2 do laboratório de eletrônica da Universidade de Brasília, que busca estudar o funcionamento de circuitos reguladores de tensão utilizando diodos.
\end{abstract}

\begin{IEEEkeywords}
diodo, diodo Zener, circuitos, semi-condutores, modelo ideal, regulador de tensão
\end{IEEEkeywords}

\section{Questões}
\begin{enumerate}
    % Questão 1
    \item {}
\end{enumerate}

\begin{thebibliography}{00}
\bibitem{sedra} A. Sedra and K. Smith, Microelectronic circuits, Oxford University press, 5ª Edição, 2004, pp. 148--153.
\bibitem{datasheet_diodo} Datasheet do diodo 1N4007. Disponível em: http://pdf1.alldatasheet.com/datasheet-pdf/view/14624/PANJIT/1N4007.html. Acesso em: 7 abril 2019
\bibitem{datasheet_diodo_zener} Datasheet do diodo 1N4739. Disponível em: http://pdf1.alldatasheet.com/datasheet-pdf/view/61857/GE/1N4739.html. Acesso em: 7 abril 2019

\end{thebibliography}

\end{document}
